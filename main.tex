\documentclass[floatfix, preprint, aps, jinst,superscriptaddress]{revtex4-1}
% \documentclass[floatfix, reprint, aps, prd,superscriptaddress]{revtex4-1}

\usepackage[english]{babel}
\usepackage[utf8]{inputenc}
\usepackage{amsmath,amssymb,amsfonts}
\usepackage{graphicx}
%\usepackage[colorinlistoftodos]{todonotes}

\usepackage{bm}
\usepackage{latexsym}
\usepackage{epsf}
\usepackage{rotating}
\usepackage{epsfig,graphics,rotate}
\usepackage{wrapfig}
%%\usepackage{subfigure}
\usepackage{array,hhline,dcolumn}%
\usepackage[normalem]{ulem}
% \usepackage[font={small,it}]{caption}
\usepackage{hyperref}
\usepackage{placeins}
\usepackage{geometry}

\usepackage{lineno}
\linenumbers


% % Bibliography stuff:
% \bibliographystyle{apsrev4-1}


% E-mail: \email{corey.adams@yale.edu, andrzej.szelc@manchester.ac.uk}}

\newcommand{\uboone}{MicroBooNE~}

% \keywords{neutrinos; liquid argon; time projection chamber}

\begin{document}


% \author{C. Adams}
% \affiliation{Harvard University}

\collaboration{The \uboone Collaboration}
\noaffiliation


\begin{abstract}
At the end of January, \uboone ramped down it's drift HV system after a series of unusual and worrying behavior on HV monitoring plots.  This document presents a summary of the tests performed, diagnostics developed, and a chronological ordering of events.  We also include some ``lessons learned'' that may be useful to future LArTPCs.  Ultimately, it was shown that the source of the high voltage instabilities was a poor connection between the high voltage feed-through and the cup connecting it to the cathode.

\end{abstract}


\title{High Voltage Diagnostics and Trouble Shooting in \uboone}

\date{\today}


\maketitle




% \tableofcontents

\section{Introduction}

The \uboone detector is a large-scale liquid argon time projection chamber \cite{detector-paper}.  It has been running since 2015, and filled with cold liquid argon for nearly two years at the time of this publication.  In late January 2017, the detector high voltage was ramped down to 0kV in response to problems that were being observed.  After four weeks of tests, diagnostics and downtime the cause of the high voltage problems was decisively found to be a poor connection between the high voltage feed-through and the cup it is inserted to that connects to the cathode.

In this document, we present a summary of the design of \uboone's high voltage system (Section~\ref{sec:HV-design}), as well as a discussion of the symptoms observed in the detector that lead to the conclusion that the high voltage system had a problem (Section~\ref{sec:symptoms}).  Further, we describe the tests we performed to narrow down on the source of the problem (Section~\ref{sec:diagnostics}) and the resolution and lessons learned from this situation (Section~\ref{sec:resolution}).  

\section{Description of \uboone HV System \label{sec:HV-design}}

This section describes the details of the high voltage supply system for \uboone, from the power supply and warm filtering components through the entire system to the detector ground.

\subsection{HV Supply \label{sec:HV-Supply}}

\subsection{HV Feedthrough \label{sec:HV-Feedthrough}}

\subsection{Cathode and Resistor Chain \label{sec:Cathode-and-Resistor-Chain}}

\begin{figure}[htbp]
  \centering
  \includegraphics[width=0.95\textwidth]{figures/Full_Circuit_revA.pdf}
  \includegraphics[width=0.95\textwidth]{figures/FINAL_Field_Cage_revA.pdf}
  \caption{caption}
  \label{fig:Field-Cage-Circuit}
\end{figure}

\subsection{Anode and Wire Bias \label{sec:Anode-and-Wire-Bias}}

\subsection{Pickoff Point \label{sec:Pickoff-Point}}

\section{Symptoms}

\subsection{Pickoff Point Instability}

\subsection{Glassman Current RMS}

\subsection{TPC Asic LV Current Draw}

\subsection{``Burst'' Events}

\section{Diagnostics}

\subsection{(Warm) HV Supply Tests}

After the weekend of Jan 27th, 2017, MicroBooNE powered down the drift high voltage supply to investigate.  To diagnose the source of the troubling, a large number of procedures were attempted, described here.

\subsubsection{Glassman HV Supply replacement}
The very first test performed on the system was to replace the high voltage power supply.  The system is a a Glassman power supply, described in Section~\ref{sec:HV-Supply}.  A spare was available and installed, and when the spare supply was powered on no observable difference was detected in the behavior of the system.  The spare supply was not ramped to a high voltage initially, since the symptoms were clearly occurring at a relatively low ( \textless 10kV) voltage.

\subsubsection{HV Cable inspection}
The next obvious step to search for a hardware failure was in the high voltage cables that bring the high voltage from the Glassman supply, through both filter pots, and into the cryostat high voltage feedthrough.  All connections of cable to supply or filters pots were checked, and no obvious defects were found.

\subsubsection{AC Power Distribution Inspection}

Considering that one main symptom of the high voltage instabilities is the high RMS on the current output from the HV power supply, we investigated the AC power distribution to the power supply itself.  Using an AC analyzer (``ONEAC ONEView Line Noise Viewing Interface'' \cite{lineviewer}), we determined that there was no excess noise on the AC input to the high voltage supply than to any other areas of the detector systems.

\subsubsection{``In air'' Test of HV Supply}

As a last test of the warm side of the high voltage supply system, we pulled the cable that connects to the cryostat high voltage feedthrough out of the feedthrough, and ramped it to 20kV in air.  No symptoms were observed during this test, indicating that the power supply, filter pots, high voltage cables, and AC power distribution were all functioning properly.  Based off of this, it was concluded that the warm side of the high voltage supply system was in good health, and the problems with the HV system were downstream of the cryostat's high voltage feedthrough.

\subsection{V vs. I Tests on cathode}

\subsection{Pickoff Point Measurements}
\subsubsection{Current-source mode measurements}
\subsubsection{Measurement of field cage resistance}
\subsubsection{Voltage-source mode measurements}
\subsubsection{Measurement of pickoff point resistance}
\subsubsection{Measurement of field cage resistance}
\subsubsection{Measurement of burst rate at pickoff point bias}

\subsection{``Burst'' Analysis}

\subsection{Cathode Pulse Tests}

\section{Resolution \label{sec:resolution}}




\input{conclusion}

% \appendix{Chronological Timeline}


% \section{Acknowledgements}


\bibliography{bibliography}

\end{document}

